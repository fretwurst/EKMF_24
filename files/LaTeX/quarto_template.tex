% Options for packages loaded elsewhere
\PassOptionsToPackage{unicode$for(hyperrefoptions)$,$hyperrefoptions$$endfor$}{hyperref}
\PassOptionsToPackage{hyphens}{url}
$if(colorlinks)$
\PassOptionsToPackage{dvipsnames,svgnames,x11names}{xcolor}
$endif$
$if(CJKmainfont)$
\PassOptionsToPackage{space}{xeCJK}
$endif$
%

$doc-class.tex()$

% \documentclass[twoside, fontsize=10pt]{scrreport}
\usepackage{amsmath,amssymb}
$if(linestretch)$
\usepackage{setspace}
$endif$
\usepackage{iftex}
\ifPDFTeX
  \usepackage[$if(fontenc)$$fontenc$$else$T1$endif$]{fontenc}
  \usepackage[utf8]{inputenc}
  \usepackage{textcomp} % provide euro and other symbols
\else % if luatex or xetex
$if(mathspec)$
  \ifXeTeX
    \usepackage{mathspec} % this also loads fontspec
  \else
    \usepackage{unicode-math} % this also loads fontspec
  \fi
$else$
  \usepackage{unicode-math}
$endif$
  \defaultfontfeatures{Scale=MatchLowercase}$-- must come before Beamer theme
  \defaultfontfeatures[\rmfamily]{Ligatures=TeX,Scale=1}
\fi
% Use upquote if available, for straight quotes in verbatim environments
\IfFileExists{upquote.sty}{\usepackage{upquote}}{}
$if(verbatim-in-note)$
\usepackage{fancyvrb}
$endif$
\usepackage{xcolor}

$if(geometry)$
\usepackage[$for(geometry)$$geometry$$sep$,$endfor$]{geometry}
$endif$

\usepackage{pdflscape}

$if(listings)$
\usepackage{listings}
\newcommand{\passthrough}[1]{#1}
\lstset{defaultdialect=[5.3]Lua}
\lstset{defaultdialect=[x86masm]Assembler}
$endif$
$if(lhs)$
\lstnewenvironment{code}{\lstset{language=Haskell,basicstyle=\small\ttfamily}}{}
$endif$
$if(svg)$
\usepackage{svg}
$endif$
$if(strikeout)$
$-- also used for underline
\ifLuaTeX
  \usepackage{luacolor}
  \usepackage[soul]{lua-ul}
\else
  \usepackage{soul}
\fi
$endif$
\setlength{\emergencystretch}{3em} % prevent overfull lines
$if(numbersections)$
\setcounter{secnumdepth}{$if(secnumdepth)$$secnumdepth$$else$5$endif$}
$else$
\setcounter{secnumdepth}{-\maxdimen} % remove section numbering
$endif$
$if(block-headings)$
% Make \paragraph and \subparagraph free-standing
\ifx\paragraph\undefined\else
  \let\oldparagraph\paragraph
  \renewcommand{\paragraph}[1]{\oldparagraph{#1}\mbox{}}
\fi
\ifx\subparagraph\undefined\else
  \let\oldsubparagraph\subparagraph
  \renewcommand{\subparagraph}[1]{\oldsubparagraph{#1}\mbox{}}
\fi

$endif$
$if(pagestyle)$
\pagestyle{$pagestyle$}
$endif$
$pandoc.tex()$
$if(lang)$
\ifLuaTeX
\usepackage[bidi=basic]{babel}
\else
\usepackage[bidi=default]{babel}
\fi
$if(babel-lang)$
\babelprovide[main,import]{$babel-lang$}
$if(mainfont)$
\ifPDFTeX
\else
\babelfont{rm}[$for(mainfontoptions)$$mainfontoptions$$sep$,$endfor$]{$mainfont$}
\fi
$endif$
$endif$
$for(babel-otherlangs)$
\babelprovide[import]{$babel-otherlangs$}
$endfor$
$for(babelfonts/pairs)$
\babelfont[$babelfonts.key$]{rm}{$babelfonts.value$}
$endfor$
% get rid of language-specific shorthands (see #6817):
\let\LanguageShortHands\languageshorthands
\def\languageshorthands#1{}
$endif$
\ifLuaTeX
  \usepackage{selnolig}  % disable illegal ligatures
\fi
$if(dir)$
\ifPDFTeX
  \TeXXeTstate=1
  \newcommand{\RL}[1]{\beginR #1\endR}
  \newcommand{\LR}[1]{\beginL #1\endL}
  \newenvironment{RTL}{\beginR}{\endR}
  \newenvironment{LTR}{\beginL}{\endL}
\fi
$endif$
$if(biblio-config)$
$if(natbib)$
\usepackage[$natbiboptions$]{natbib}
\bibliographystyle{$if(biblio-style)$$biblio-style$$else$plainnat$endif$}
$endif$
$if(biblatex)$
\usepackage[$if(biblio-style)$style=$biblio-style$,$endif$$for(biblatexoptions)$$biblatexoptions$$sep$,$endfor$]{biblatex}
$for(bibliography)$
\addbibresource{$bibliography$}
$endfor$
$endif$
$endif$
$if(nocite-ids)$
\nocite{$for(nocite-ids)$$it$$sep$, $endfor$}
$endif$
$if(csquotes)$
\usepackage{csquotes}
$endif$
\IfFileExists{bookmark.sty}{\usepackage{bookmark}}{\usepackage{hyperref}}
\IfFileExists{xurl.sty}{\usepackage{xurl}}{} % add URL line breaks if available
\urlstyle{same} % disable monospaced font for URLs
$if(links-as-notes)$
% Make links footnotes instead of hotlinks:
\DeclareRobustCommand{\href}[2]{#2\footnote{\url{#1}}}
$endif$
$if(verbatim-in-note)$
\VerbatimFootnotes % allow verbatim text in footnotes
$endif$


\hypersetup{
 pdflang = de, colorlinks = true, linkcolor = black, urlcolor = black,
 pdfauthor={$for(author)$ $author.name$, $author.given-names$ $endfor$},
    pdfsubject={Absolvierendenbefragung des BfS für das IKMZ},
    pdftitle={Absolvierendenbefragung des BfS für das IKMZ},
    pdfkeywords={Befragung, Absolvierende, Alumni, IKMZ, Kommunikationswissenschaft}
   lang = {DE}
}

\usepackage{scrlayer-scrpage}

\definecolor{headergray}{RGB}{192,192,192}

\addtokomafont{disposition}{\sffamily}
\renewcommand{\familydefault}{\sfdefault}

\title{}
\date{}

\setlength{\abovecaptionskip}{10pt plus 3pt minus 2pt}
\setlength{\belowcaptionskip}{12pt plus 3pt minus 2pt}

\captionsetup{justification=raggedright, singlelinecheck=false}

%%%%%%%%%%%%%%%%%%%%%%%%%%%%%%%%%%%%%%%%%%%%

              %Dokumentstart

%%%%%%%%%%%%%%%%%%%%%%%%%%%%%%%%%%%%%%%%%%%%


\begin{document}

\pagestyle{scrheadings}
\clearscrheadfoot
\ohead{\textcolor{headergray}{\pagemark}}

\automark[section]{chapter}
\ihead{\textcolor{headergray}{\headmark}}

\ofoot{}
\ifoot{}

%\cfoot[]{$title$} %unten Mitte
%\setheadsepline{.2pt}



\begin{titlepage}
\sffamily

\setlength\parindent{0pt}

\hfill \includegraphics[width = 6cm]{files/LaTeX/uzh_logo_d_pos.pdf}\par

\vspace{2cm}


{\bfseries\Huge $title$ \\[1cm] } % Den Zeilenumbruch immer ins Huge mit rein, damit der Zeilenabstand auch angepasst wird

$if(subtitle)$ {\bfseries\Large $subtitle$ \\[1cm]} $endif$

{\bfseries \Large
$for(author)$ $author.given-names$ $author.surname$ $sep$ und $endfor$ \\ [1cm]
}


$if(Version)$ 
\vfill
{\bfseries \Large $Version$  \\[1cm] } 
$endif$

\vfill 

Zuletzt aktualisiert: $date$

\vfill

\begin{tabbing}
Kontakt:\\
Dr. Benjamin Fretwurst\\
Institut für Kommunikationswissenschaft und Medienforschung -- IKMZ\\
Andreasstrasse 15\\
8050 Zürich\\
b.fretwurst@ikmz.ch\\[0.3cm]
\end{tabbing}

\setlength\parindent{1em}

\end{titlepage}

\makeatletter
\AddToShipoutPicture{\setlength{\unitlength}{1cm}\put(24.32,18.65){{\includegraphics[height=.6cm]{files/LaTeX/uzh_logo_d_pos.pdf}}}}
\makeatletter

\cleardoublepage

%Überschriften unterdrücken durch IV und TV (wegen multicolumn)
\renewcommand{\listoftables}{\@starttoc{lot}}
\renewcommand{\tableofcontents}{\@starttoc{toc}}
\renewcommand{\listoffigures}{\@starttoc{lof}}

%Mehr Platz für breite Tabellennummern
\renewcommand{\l@table}{\@dottedtocline{1}{1em}{3em}}
\makeatother

\section*{Inhalt}
\label{sec:inhalt}
\pdfbookmark[1]{\contentsname}{toc}
%  \begin{multicols}{2}
    \tableofcontents
%  \end{multicols}


% \pdfbookmark[1]{Tabellen/Abbildungen}{lot}
\subsection*{Tabellenverzeichnis}
\label{sec:tabellenverzeichnis}

%\begin{multicols}{2}
\listoftables
%\end{multicols}

\subsection*{Abbildungsverzeichnis}
\label{sec:Abbildungsverzeichnis}

%\begin{multicols}{2}
\listoffigures
%\end{multicols}

\clearpage

% $before-body.tex()$
% $for(include-before)$
% $include-before$
% $endfor$

$if(linestretch)$
\setstretch{$linestretch$}
$endif$

\renewcommand{\familydefault}{\sfdefault}

\setlength{\parindent}{1em}

$body$

$before-bib.tex()$

% $if(has-frontmatter)$
% \backmatter
% $endif$
$biblio.tex()$

$for(include-after)$
$include-after$

$endfor$
$after-body.tex()$
\end{document}
